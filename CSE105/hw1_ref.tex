\documentclass[letterpaper,12pt]{article}
\usepackage{fullpage}
\usepackage{amssymb}
\usepackage{mathtools}
\usepackage[hyphens]{url}
\usepackage[breaklinks,linkcolor=black,citecolor=black,urlcolor=black,bookmarks=true,bookmarksopen=false,pdftex]{hyperref}
\usepackage{graphicx}
\usepackage{color}
\usepackage[protrusion=true,expansion=true,kerning]{microtype}

\setlength{\topmargin}{0in}
\setlength{\headheight}{0in}
\setlength{\headsep}{0in}
\setlength{\topskip}{0in}

\newcommand{\Z}{\mathbb{Z}}
\newcommand{\F}{\mathbb{F}}
\newcommand{\R}{\mathbb{R}}

\def\dash---{\kern.16667em---\penalty\exhyphenpenalty\hskip.16667em\relax}

\begin{document}

\newlength{\boxwidth}
\setlength{\boxwidth}{\textwidth}
\addtolength{\boxwidth}{-2cm}
\noindent\framebox[\textwidth]{\hfil
\parbox[t]{\boxwidth}{%
{\bfseries CSE\,105: Automata and Computability Theory \hfill Autumn 2014}
\begin{center}\huge Homework \#1\end{center}
Due: Tuesday, October 14th, 2014, 11:59 \textsc{pm}%
}\hfil}
\vspace{0.7cm}

\begin{description}

\item[Problem 1] Sign up for an account on Automata Tutor, at
  \url{http://automatatutor.com}.  If you are a UCSD student, use your
  official UCSD e-mail for your account, so that we know whom to
  assign credit to.  If you are an extension student, use the same
  e-mail address you gave us to use for GradeSource.  Enroll in
  CSE~105's Automata Tutor section, which has Course~ID
  ``\texttt{28CSE105(}'' and password ``\texttt{3A4F0ZUB}''.

  You may practice designing automata using the practice problem
  sets.

  \begin{description}
  \item[a--e.] Complete the five problems that constitute CSE~105's
    HW1 on Automata Tutor.  For each problem, you will have three
    chances.
  \end{description}

\item[Problem 2] Closure properties.
  For a positive integer~$m$, we define the \emph{principal} of~$m$,
  denoted $P(m)$, as the set $\{im \mid i \in \Z\}$.  For example,
  $P(1)$~is just~$\Z$, and $P(2)$~is the even numbers.
  \begin{description}
  \item[a.] Prove that, for any positive integer~$m$, the set $P(m)$
    is closed under addition.  (I.e., show that $a,b \in P(m)$ implies
    $a+b \in P(m)$.)
  \item[b.] Prove that, for any positive integer~$m$, the set~$P(m)$
    is closed under multiplication with the set~$\Z$.  (I.e., show
    that $a \in P(m)$, $b \in \Z$ implies $ab \in P(m)$.)
  \item[c.] Prove that the set (of sets) $\mathfrak{P} = \bigl\{ P(m)
    \bigm| m \in \Z^+ \bigr\}$ is closed under addition.  (I.e., show
    that $A, B \in \mathfrak{P}$ implies $A+B \in \mathfrak{P}$.  Note
    that here ``$A+B$'' is set addition, i.e., $\{a+b \mid a \in A, b
    \in B \}$.)
  \end{description}
  For part c., you may wish to use the fact that if $r$ is the gcd of
  $m$~and $n$ then there exist integers $u$~and $v$ such that $um + vn
  = r$.
\item[Problem 3] Haskell.

  Install the Haskell Platform, either 2013.2 or 2014.2.  Read ``Notes
  on Haskell,'' linked from the course Website.  Make sure that you
  can load \texttt{DFA.hs} (also linked from the course Website) in
  the GHCI interpreter.

  For this problem, you will turn in a file, \texttt{hw1p3.hs}, that
  defines a haskell variable \texttt{machine} that holds a 5-tuple DFA
  for the following language:

  \begin{quotation}
    For the alphabet~$\{\texttt{0},\texttt{1}\}$, describe a DFA that
    recognizes the language of all strings that, interpreted as
    numbers in big-endian binary notation (i.e., with least
    significant bit last), are divisible by 3.  In big-endian binary
    notation, 11~is encoded as \texttt{1011}, 31337~as
    \texttt{111101001101001}.  (It is okay for numbers to be
    represented with leading zeroes.)

    \textbf{Hint:} Suppose the number represented by $a_1 a_2 \cdots
    a_{k}$ in big-endian binary has residue $x$~mod~3.  What residue
    mod~3 does the number $a_1 a_2 \cdots a_{k} a_{k+1}$ have, as a
    function of $x$~and the binary digit~$a_{k+1}$?

    \textbf{Hint:} You need just three states.
  \end{quotation}

  (Note: you should define this machine manually, not use the JFLAP
  import utility code.)

  On the course Website, you can obtain a starter \texttt{hw1p3.hs}
  file that imports the appropriate modules; your job is to fill in
  the definition of the variable \texttt{machine} in that file.

  You can test that your machine behaves correctly using
  \texttt{evalDFA} (defined in \texttt{DFA.hs}) and the utility
  function \texttt{intToBinary}, for converting a number to a
  big-endian binary string, which is included in the starter file.

  The starter file also includes a wrapper, called \texttt{test},
  around these two functions, that converts its numeric argument to a
  string (in big-endian binary format) and calls \texttt{evalDFA} on
  the DFA in \texttt{machine} with the resulting string.  If your
  machine is correct, \texttt{test} should evaluate to \texttt{True}
  exactly when given an argument that is divisible by 3.


\end{description}
\end{document}
